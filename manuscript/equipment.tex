\chapter{Equipment}

\section{About Equipment}

This chapter contains information and statistics for all kinds
of items, weapons, armor, and other useful things. All costs
are listed in coppers, which is the standard coin of Yrda.

Items that make use of Resource Dice are denoted with an (R) in their name.

\section{Currency}

The coinage of the realm is the "copper," a small coin made of copper.
As a single copper may be worth more than goods of low value, it's a
common practice to split coppers into eight equal pieces. These pieces
are just called "shards." Sometimes people will refer to mixed prices as
(for example) "twelve and two," meaning 12 coppers and 2 shards.

In the following equipment lists, coppers are abbreviated "c" and shards are
abbreviated "s."

\section{Metals Available}

Yrda has not yet discovered the secret of working iron. The most common metal used
for weapons and armor is bronze, which is a mixture of copper and tin. Gold and
silver are used for decoration and jewelry, but are hard to come by.

\section{How Much Can I Carry?}

Be reasonable about what your character can carry. A real person
couldn't carry twelve different weapons and 300 lbs of miscellaneous
gear in a small backpack.\index{carrying capacity}

In general, keep close to the following limits:

\begin{itemize}
  \item one weapon wielded and one weapon stowed
  \item a shield OR a second weapon wielded
  \item fifty pounds of other small gear stored in a backpack
  \item a few ounces of coins and other light items in belt pouches
\end{itemize}

Anything more than that will require a mount, wagon, or other
additional storage space.

Equipment that you don't have on you will probably be stored
back at an inn, home, camp, or other base of operations.

\section{What Armor Can I Wear?}

You can wear a helmet and body armor. A shield is held in your
off hand, if you choose to use one.\index{armor limits}

\section{Random Equipment Tables}

The following tables are used during character creation. When told
to roll on one of these tables, roll the specified dice once.

\bottomcaption{Random weapons}
\tablefirsthead{\hline \multicolumn{1}{|c|}{\textbf{Roll}} &
                       \multicolumn{1}{c|}{\textbf{Item(s)}} \\ \hline }
\begin{center}
{\rowcolors{3}{white}{light-gray}
\begin{xtabular}{|l|l|l|}
1 & Dagger \\
2 & Short sword \\
3 & Short sword and buckler \\
4 & Battle axe \\
5 & Spear \\
6 & Warhammer OR long sword \\
7 & Short sword and strapped shield \\
8 & Long sword and strapped shield \\
9 & Battle axe and strapped shield \\
10 & Spear, short sword, and strapped shield \\
\hline
\end{xtabular}
}
\end{center}\index{random weapons}

\bottomcaption{Random armor}
\tablefirsthead{\hline \multicolumn{1}{|c|}{\textbf{Roll}} &
                       \multicolumn{1}{c|}{\textbf{Item(s)}} \\ \hline }
\begin{center}
{\rowcolors{3}{white}{light-gray}
\begin{xtabular}{|l|l|l|}
1 & Boiled leather \\
2 & Boiled leather and horned helmet \\
3 & Brigandine \\
4 & Brigandine and horned helmet \\
5 & Chainmail \\
6 & Chainmail and horned helmet \\
7 & Chainmail and kettle helm OR nasal helm \\
8 & Plate armor \\
9 & Plate armor and horned helmet \\
10 & Plate armor and great helm OR nasal helm \\
\hline
\end{xtabular}
}
\end{center}\index{random armor}

\section{Equipment Lists}

The following lists show the details of various kinds of equipment.

\subsection{Body Armor}

Body armor protects your torso, and optionally your shoulders,
arms, and legs.

\bottomcaption{List of body armor}
\tablefirsthead{\hline \multicolumn{1}{|c|}{\textbf{Name}} &
                       \multicolumn{1}{c|}{\textbf{AP}} &
                       \multicolumn{1}{c|}{\textbf{Cost}} \\ \hline }
\begin{center}
{\rowcolors{3}{white}{light-gray}
\begin{xtabular}{|l|l|l|}
Boiled leather & 1 & 20c \\
Brigandine & 2 & 40c \\
Chainmail & 3 & 80c \\
Plate armor & 4 & 200c \\
\hline
\end{xtabular}
}
\end{center}

\subsection{Helmets}

Helmets cover part or all of your head. They come in a variety
of materials and levels of workmanship.

\bottomcaption{List of helmets}
\tablefirsthead{\hline \multicolumn{1}{|c|}{\textbf{Name}} &
                       \multicolumn{1}{c|}{\textbf{AP}} &
                       \multicolumn{1}{c|}{\textbf{Cost}} \\ \hline }
\begin{center}
{\rowcolors{3}{white}{light-gray}
\begin{xtabular}{|l|l|l|}
Nasal helm & 2 & 13c \\
Great helm & 2 & 20c \\
Kettle helm & 2 & 15c \\
Horned helmet & 1 & 11c \\
Wolf's-head helm & 1 & 18c \\
\hline
\end{xtabular}
}
\end{center}

\subsection{Shields}

Shields come in three types - bucklers, strapped shields, and
tower shields. Bucklers are strapped to your arm and are not
held, freeing up a hand but lacking the defense of a larger
shield. Strapped shields have a handle or strap and are held
in your off hand. Tower shields are heavy and huge, and are
meant as mobile defensive structures. Tower shields only offer
cover and do not grant Armor Points. Moving a tower shield to
a new position is a minor action.

\bottomcaption{List of bucklers}
\tablefirsthead{\hline \multicolumn{1}{|c|}{\textbf{Name}} &
                       \multicolumn{1}{c|}{\textbf{AP}} &
                       \multicolumn{1}{c|}{\textbf{Cost}} \\ \hline }
\begin{center}
{\rowcolors{3}{white}{light-gray}
\begin{xtabular}{|l|l|l|}
Round & 1 & 13c \\
Oval & 1 & 20c \\
Rectangular & 1 & 7c \\
\hline
\end{xtabular}
}
\end{center}

\bottomcaption{List of strapped shields}
\tablefirsthead{\hline \multicolumn{1}{|c|}{\textbf{Name}} &
                       \multicolumn{1}{c|}{\textbf{AP}} &
                       \multicolumn{1}{c|}{\textbf{Cost}} \\ \hline }
\begin{center}
{\rowcolors{3}{white}{light-gray}
\begin{xtabular}{|l|l|l|}
Kite & 2 & 25c \\
Heater & 2 & 20c \\
Targe & 2 & 15c \\
\hline
\end{xtabular}
}
\end{center}

\bottomcaption{List of tower shields}
\tablefirsthead{\hline \multicolumn{1}{|c|}{\textbf{Name}} &
                       \multicolumn{1}{c|}{\textbf{Cover}} &
                       \multicolumn{1}{c|}{\textbf{Cost}} \\ \hline }
\begin{center}
{\rowcolors{3}{white}{light-gray}
\begin{xtabular}{|l|l|l|}
Pavise & Half & 30c \\
Mantlet & Full & 25c \\
\hline
\end{xtabular}
}
\end{center}

\subsection{Melee Weapons}

A wide variety of melee weapons are in use in Yrda. Different
cultures favor different weapons, though short swords are the
most common.

\bottomcaption{List of melee weapons}
\tablefirsthead{\hline \multicolumn{1}{|c|}{\textbf{Name}} &
                       \multicolumn{1}{c|}{\textbf{Damage}} &
                       \multicolumn{1}{c|}{\textbf{Cost}} \\ \hline }
\begin{center}
{\rowcolors{3}{white}{light-gray}
\begin{xtabular}{|l|l|l|}
Battle Axe & 1d8 & 20c \\
Dagger & 1d4 & 2c \\
Knife & 1d4 & 2c \\
Lance & 1d6 & 15c \\
Long Sword & 1d6 & 12c \\
Mace & 1d6 & 5c \\
Morningstar & 1d6 & 10c \\
Scythe & 1d8 & 10c \\
Short Sword & 1d6 & 8c \\
Spear & 1d8 & 10c \\
Trident & 1d8 & 20c \\
Warhammer & 1d6 & 5c \\
Whip & 1d4 & 2c \\
\hline
\end{xtabular}
}
\end{center}

\subsection{Ranged Weapons}

The most common ranged weapon in this era is the short bow. Long bows
are difficult to draw and longtime wielders end up suffering deformities
of the spine. Crossbows are easier to fire, but take longer to set up for
a shot. They are best used in pairs by two people - one to load and set,
the other to fire, trading between the two. Commoners often use slings or
throwing knives.

\bottomcaption{List of ranged weapons}
\tablefirsthead{\hline \multicolumn{1}{|c|}{\textbf{Name}} &
                       \multicolumn{1}{c|}{\textbf{Damage}} &
                       \multicolumn{1}{c|}{\textbf{Range (Close/Mid/Long, in feet)}} &
                       \multicolumn{1}{c|}{\textbf{Cost}} \\ \hline }
\begin{center}
{\rowcolors{3}{white}{light-gray}
\begin{xtabular}{|l|l|l|l|}
Short bow & 1d6 & 20/100/300 & 20c \\
Long bow & 1d10 & 40/200/600 & 100c \\
Crossbow & 1d6 & 20/100/300 & 50c \\
Bolas & 1d4 & 10/20/50 & 5s \\
Sling & 1d4 & 10/20/50 & 5s \\
Throwing Knife & 1d4 & 10/20/50 & 5s \\
\hline
\end{xtabular}
}
\end{center}

\bottomcaption{List of ammunition for ranged weapons}
\tablefirsthead{\hline \multicolumn{1}{|c|}{\textbf{Name}} &
                       \multicolumn{1}{c|}{\textbf{Cost}} \\ \hline }
\begin{center}
{\rowcolors{3}{white}{light-gray}
\begin{xtabular}{|l|l|}
Arrow (R) & 1s \\
Bolt, crossbow (R) & 2s \\
\hline
\end{xtabular}
}
\end{center}

\subsection{Food and Drink}

This is a short list of food and drink prices for common items.
Unless marked with an (R), these prices are for one-off items in
a tavern or similar situation.

\bottomcaption{List of food and drink}
\tablefirsthead{\hline \multicolumn{1}{|c|}{\textbf{Name}} &
                       \multicolumn{1}{c|}{\textbf{Cost}} \\ \hline }
\begin{center}
{\rowcolors{3}{white}{light-gray}
\begin{xtabular}{|l|l|}
Ale, cup & 1s \\
Ale, pint & 2s \\
Ale (R) & 4s \\
Beef, roast & 2s \\
Beer, cup & 1s \\
Beer, pint & 2s \\
Beer (R) & 4s \\
Bread & 1s \\
Gruel, bowl & 1s \\
Mead, cup & 1s \\
Mead (R) & 4s \\
Mutton, roast & 2s \\
Rice wine, cup & 2s \\
Travel rations (R) & 1c \\
Water, bottle & 2s \\
Wine, cup & 2s \\
Wine (R) & 6s \\
\hline
\end{xtabular}
}
\end{center}

\subsection{Traveling Gear}

This list is for items that would be of use in traveling.

\bottomcaption{List of traveling gear}
\tablefirsthead{\hline \multicolumn{1}{|c|}{\textbf{Name}} &
                       \multicolumn{1}{c|}{\textbf{Cost}} \\ \hline }
\begin{center}
{\rowcolors{3}{white}{light-gray}
\begin{xtabular}{|l|l|}
Backpack & 2s \\
Bedroll & 1s \\
Blanket & 1s \\
Box, large & 2s \\
Box, small & 1s \\
Candle, 1 day & 1s \\
Chain, 10 feet & 5s \\
Chest, large & 1c \\
Chest, small & 5s \\
Fishing hook & 1s \\
Fishing line & 1s \\
Fishing net & 1s \\
Fishing pole & 1s \\
Flask, empty & 1s \\
Kettle & 2s \\
Lantern & 5s \\
Lantern oil, 1 day & 1s \\
Pot, iron & 2s \\
Sack, large & 2s \\
Sack, small & 1s \\
Satchel & 1s \\
Tent, 2 person & 1c \\
Tent, 4 person & 2c \\
Torches (R) & 1s \\
Waterskin & 1s \\
\hline
\end{xtabular}
}
\end{center}

\subsection{Dungeoneering Gear}

This list is for items that would be of use in a dungeon.

\bottomcaption{List of dungeoneering gear}
\tablefirsthead{\hline \multicolumn{1}{|c|}{\textbf{Name}} &
                       \multicolumn{1}{c|}{\textbf{Cost}} \\ \hline }
\begin{center}
{\rowcolors{3}{white}{light-gray}
\begin{xtabular}{|l|l|}
Caltrops (R) & 1s \\
Glue, 1 pint & 3s \\
Hacksaw & 6s \\
Hammer & 1s \\
Ladder, 10 feet & 5s \\
Oil, 1 pint & 1s \\
Pick, iron & 1c \\
Pitons, 4 & 1s \\
Rope ladder, 10 feet & 3s \\
Rope, 50 feet & 4s \\
Saw, iron & 1c \\
Shovel & 2s \\
Tinderbox & 5s \\
Wax, 1 pound & 1s \\
\hline
\end{xtabular}
}
\end{center}

\subsection{Clothing}

Common garments are listed here. All cultures have their own modes of dress and this price list should be considered a rough guide.

\bottomcaption{List of clothing}
\tablefirsthead{\hline \multicolumn{1}{|c|}{\textbf{Name}} &
                       \multicolumn{1}{c|}{\textbf{Cost}} \\ \hline }
\begin{center}
{\rowcolors{3}{white}{light-gray}
\begin{xtabular}{|l|l|}
Belt, leather & 1s \\
Boots, leather & 2s \\
Boots, long & 2s \\
Cloak, wool & 1s \\
Dress, silk & 2c \\
Dress, wool & 1s \\
Gloves, leather & 4s \\
Gloves, riding & 3s \\
Hat, felt & 1s \\
Hat, leather & 1s \\
Pants, wool & 1s \\
Robe, wool & 1s \\
Tunic, linen & 1s \\
Tunic, silk & 1c \\
\hline
\end{xtabular}
}
\end{center}

\subsection{Musical Instruments}

Different cultures have different instruments. This list is for the most common.

\bottomcaption{List of musical instruments}
\tablefirsthead{\hline \multicolumn{1}{|c|}{\textbf{Name}} &
                       \multicolumn{1}{c|}{\textbf{Cost}} \\ \hline }
\begin{center}
{\rowcolors{3}{white}{light-gray}
\begin{xtabular}{|l|l|}
Bagpipes, double drone & 2c \\
Bagpipes, single drone & 1c \\
Bell, brass & 5s \\
Bell, silver & 1c \\
Flute, silver & 2c \\
Flute, wooden & 1c \\
Hand drum & 2s \\
Harmonica & 5s \\
Harp & 3c \\
Lute & 1c \\
Lyre & 3c \\
Pan flute & 5s \\
Sitar & 2c \\
Xylophone & 2c \\
Zither & 2c \\
\hline
\end{xtabular}
}
\end{center}

\subsection{Mounts}

This list is for mounts that can be purchased. Different biomes will have different mounts available.

\bottomcaption{List of mounts}
\tablefirsthead{\hline \multicolumn{1}{|c|}{\textbf{Name}} &
                       \multicolumn{1}{c|}{\textbf{Cost}} \\ \hline }
\begin{center}
{\rowcolors{3}{white}{light-gray}
\begin{xtabular}{|l|l|}
Camel & 4c \\
Drake, riding & 8c \\
Horse, race & 5c \\
Horse, riding & 4c \\
Horse, war & 8c \\
Horse, work & 2c \\
Mule & 8s \\
Pony & 2c \\
Triceratops, war & 20c \\
Triceratops, work & 15c \\
\hline
\end{xtabular}
}
\end{center}

\subsection{Carts and Wagons}

This list is for common carts, wagons, and other things meant for transporting goods.

\bottomcaption{List of carts and wagons}
\tablefirsthead{\hline \multicolumn{1}{|c|}{\textbf{Name}} &
                       \multicolumn{1}{c|}{\textbf{Cost}} \\ \hline }
\begin{center}
{\rowcolors{3}{white}{light-gray}
\begin{xtabular}{|l|l|}
Cart, 2 wheel & 1c \\
Cart, 4 wheel & 2c \\
Wagon, uncovered & 2c \\
Wagon, covered & 3c \\
\hline
\end{xtabular}
}
\end{center}

\subsection{Ships}

Seagoing and river-going ships are listed here. Passage on vessels varies by distance and speed,
but can usually be had for 1s per mile per day.

\bottomcaption{List of ships}
\tablefirsthead{\hline \multicolumn{1}{|c|}{\textbf{Name}} &
                       \multicolumn{1}{c|}{\textbf{Cost}} \\ \hline }
\begin{center}
{\rowcolors{3}{white}{light-gray}
\begin{xtabular}{|l|l|}
Canoe & 5c \\
Caravel & 10,000c \\
Fishing boat & 100c \\
Frigate & 40,000c \\
Galleon & 50,000c \\
Longship & 4,000c \\
Schooner & 7,500c \\
Sloop & 5,000c \\
\hline
\end{xtabular}
}
\end{center}
