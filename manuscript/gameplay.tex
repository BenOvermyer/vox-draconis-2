\chapter{How to Play}

\begin{multicols}{2}

\section{General Gameplay}

The flow of gameplay runs in turns. The Game Master will
describe a scene, and then players take turns describing
what they do. In combat, turns are more structured, and each
enemy has a separate turn.

\section{Turn Order}

Outside of combat, turns can occur in whatever order fits the
scene.\index{turn order, noncombat}

In combat, there are four phases in which participants\index{turn order, combat}
act. Players can choose either to take a Fast Turn\index{fast turn} or a Slow
Turn\index{slow turn}. Characters controlled by the GM (called Non-Player
Characters, or NPCs) have the same choice. Turns then play out
in the following order:

\begin{enumerate}
  \item Player Fast Turn
  \item NPC Fast Turn
  \item Player Slow Turn
  \item NPC Slow Turn
\end{enumerate}

\subsection{Fast Turn}

A Fast Turn allows the character to take one major action OR
one minor action.

\subsection{Slow Turn}

A Slow Turn allows the character to take one major action and
one minor action, OR two minor actions.

\section{Actions}

There are two types of action\index{action}: Major and minor. Major actions
include casting a spell, attacking, setting a trap, or any other
action that takes the character's full attention. A minor action
includes movement, swapping equipped weapons, or other things
that the character can do while doing major actions.

Most of the time, a major action requires an "attribute roll."\index{attribute roll} The
player rolls a twenty-sided die and adds the relevant attribute
score. There may be other modifiers. The result is compared to
a target number that the GM specifies from the following:

\bottomcaption{Target numbers based on difficulty}
\tablefirsthead{\hline \multicolumn{1}{|c|}{\textbf{Difficulty}} &
                       \multicolumn{1}{c|}{\textbf{Target Number}} \\ \hline }
\begin{center}
{\rowcolors{3}{white}{light-gray}
\begin{xtabular}{|l|l|}
Very Easy & 7 \\
Easy & 10 \\
Moderate & 12 \\
Hard & 15 \\
Very Hard & 17 \\
\hline
\end{xtabular}
}
\end{center}

$$1d20 + attribute$$

\subsection{Skills and Skill Rolls}

Many actions in Vox Draconis are covered by simple attribute
rolls. However, some actions are covered by skills instead.
If an action falls under the area described by a skill\index{skill}, a
character attempting that action must do a skill roll.

A skill roll is an attribute roll, but may also include a "skill
proficiency" modifier. This adds +1 to the skill roll.

Sometimes, a skill proficiency is required to even attempt a
particular action. For example, a character would need a skill
proficiency in metallurgy to produce an alloy.

Skill proficiencies are granted by classes in character creation.
They may also be learned through the course of gameplay. See
the Developments chapter for further information.

\section{Crisis Rolls}

If a player ever gets into a situation that would have dire
consequences for them, they may make a Crisis Roll\index{crisis roll} to avoid it.
They roll a twenty-sided die, and if the roll is over 10, they
succeed. If they fail, they must spend a Crisis Point to avoid
their fate. If they don't have any Crisis Points to spend, They
suffer the consequences they were trying to avoid.

A Crisis Roll may be used for such things as avoiding death or
mitigating the results of a social engagement.

\section{The Passage of Time}

In combat, a turn lasts roughly 5 seconds of in-game time. Outside
of combat, a turn takes as much or as little in-game time as the
scene demands.\index{turn length}

\section{Combat}

The following rules govern how combat works.

\subsection{Attacking and Defending}

A character's Defense score is equal to 10 plus the sum of Armor Points they
have. If they're not wearing armor, add their Physical Dexterity.\index{defense score}

$$10 + Body AP + Helmet AP + Shield AP$$

or, for unarmored characters:

$$10 + PD$$

When a character attacks, they roll a twenty-sided die and add their Physical
Dexterity. If the result is higher than the defender's Defense score, the attack hits.\index{attacking}

$$1d20 + PD$$

\subsection{Cover}

If the target of an attack is behind cover, the chance to hit them is reduced.
An attacker's attack roll is reduced by the cover penalty listed in the table
below.\index{cover}

\bottomcaption{Cover penalties}
\tablefirsthead{\hline \multicolumn{1}{|c|}{\textbf{Amount of Cover}} &
                       \multicolumn{1}{c|}{\textbf{Penalty}} \\ \hline }
\begin{center}
{\rowcolors{3}{white}{light-gray}
\begin{xtabular}{|l|l|}
Fully hidden & -5 \\
Half hidden & -3 \\
Partly hidden & -1 \\
\hline
\end{xtabular}
}
\end{center}

\subsection{Damage}

A successful hit deals damage equal to the amount specified by
the weapon. Melee weapon damage is modified by the attacker's Physical Strength.\index{damage}

\section{Death}

If a character's Life Points reach zero or less, they fall unconscious. When this happens,
the character must make a Crisis Roll. If they fail, they must spend a Crisis Point to stay
alive. If the roll is successful or they spend the Crisis Point, they don't need to continue
rolling, and the character stabilizes but remains unconscious.\index{death}

\section{Social Engagements}

A social engagement is a conversation of importance between two or more sides. Before it begins,
divide the participants into sides. Every participant will argue for their side. Each side
determines what they're arguing for before the engagement begins.

Each participant has a turn. On their turn, they may choose to take one of the
following actions, or give up their turn for the round. Every social engagement
lasts for three rounds. At the end of the three rounds, the total points of each
side is compared, and the highest wins. In the event of a tie, there is no
victor, and the engagement is a draw.\index{social engagement}

No side can go below zero points.

Once the engagement is over, the players as a group determine the result of the winning argument.

\subsection{Actions in Social Engagements}

There are three types of actions a participant can take in a social engagement.
These are Persuade, Dissuade, or Riposte. The target number for Persuade and
Dissuade is set by the GM based on how difficult the point they're trying to
make is. Use the target number table from the Actions section.

\subsubsection{Persuade}

This is an attempt to convince another participant of the correctness of your argument. Roll a
twenty-sided die and add your Social Strength. The target number is modified by the other
participant's Social Endurance.

$$1d20 + SS$$

\subsubsection{Dissuade}

You can try and reduce the points of another participant's side. Roll a twenty-sided die
and add your Social Dexterity. The target number is modified by the other participant's
Social Dexterity. If you win, the other side loses a point.

$$1d20 + SD$$

\subsubsection{Riposte}

Instead of taking a direct action, you can complicate the result of the entire engagement
by making a Riposte. Roll a twenty-sided die and add your Social Dexterity. If the result
is higher than 10, then regardless of the result of the social engagement, there will be
a complication. The GM chooses the complication.

$$1d20 + SD$$

\section{Experience Points}

This is the reward that allows player characters to grow in ability. The Game Master gives
out Experience Points at the end of a session based on the players' actions during the
session. Experience Points can be awarded for any action by the player characters that
has a noticeable impact on the game world. See the Character Advancement chapter for
more details.\index{experience points}

\section{Player Points}

While characters can gain in power as a reward for their active participation in the game
world, players should also gain a reward for improving the game experience for everyone.
The reward for this is Player Points. Any player at the table, not just the Game Master,
can give these out. However, only one can be received by each player per session.

Player Points can be spent to alter or enhance the player character's backstory, appearance,
or other aspects of the character outside of their game mechanics. The Game Master should
use their discretion in determining the cost of such changes.\index{player points}

See the Character Advancement chapter for more details.

\end{multicols}
