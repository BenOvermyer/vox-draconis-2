\chapter{How to Play}

\begin{multicols}{2}

\section{General Gameplay}

One player acts as the Game Master and describes each scene for the others.

The others each control a character and describe what their characters do
in response to the Game Master's description of each scene.

The game is divided into "scenes." Each scene\index{scene} is either a combat,
a social engagement, exploration of a specific area, or
travel. Each scene opens with the Game Master describing the setting
and important details and then asking the players for their actions.

The flow of gameplay runs in turns. The Game Master will
describe a scene, and then players take turns describing
what they do. In combat, turns are more structured, and each
enemy has a separate turn.

\section{Maps}

There are two kinds of maps in Vox Draconis: the world map and the
scene map. The world map is made up of hexes, with each hex being
a 3 mile area. Each hex will usually have at least one thing of
interest, whether that's a town, dungeon, landmark, or something else.

The Campaigns chapter goes into more detail on building a world map for
play in Vox Draconis.

The second kind of map is the scene map. This is more flexible in scale than
a world map and acts mostly as a reference for scenes. It can be as basic
as a sketch on paper to as detailed as a full color professionally printed map.
Its purpose is solely to orient the players in a scene and help them visualize
the action.

\section{Turn Order}

Outside of combat, turns can occur in whatever order fits the
scene.\index{turn order, noncombat}

In combat, turns are based on Initiative. Each participant in combat
has an Initiative score, and they act in order of descending score.

\section{Actions}

There are two types of action\index{action}: Major and minor. Major actions
include attacking, setting a trap, casting a spell, or any other
action that takes the character's full attention. A minor action
includes movement, swapping equipped weapons, or other things
that the character can do while doing major actions.

Most of the time, a major action requires an "attribute roll."\index{attribute roll} The
player rolls a twenty-sided die and adds the relevant attribute
score. There may be other modifiers. The result is compared to
a target number that the GM specifies from the following:

\bottomcaption{Target numbers based on difficulty}
\tablefirsthead{\hline \multicolumn{1}{|c|}{\textbf{Difficulty}} &
                       \multicolumn{1}{c|}{\textbf{Target Number}} \\ \hline }
\begin{center}
{\rowcolors{3}{white}{light-gray}
\begin{xtabular}{|l|l|}
Very Easy & 7 \\
Easy & 10 \\
Moderate & 12 \\
Hard & 15 \\
Very Hard & 17 \\
\hline
\end{xtabular}
}
\end{center}

$$1d20 + attribute$$

If the roll is equal to or higher than the target number, the action
succeeds.

If the roll is more than 5 higher than the target number, the action
not only succeeds, but the character can optionally add a "flourish"
to the result. Flourishes are up to the player to describe, but they
should be something that adds to the scene in a meaningful way. \index{flourish}

If the roll is less than the target number, the action fails. The GM
describes the consequences of the failure.

If the roll is at least 5 less than the target number, the action fails
and the GM adds a "complication" to the scene. This is something that
makes the situation more difficult for the player characters. \index{complication}

\subsection{Narrative Fiat}

If a player really wants to succeed at an action, they can automatically
succeed without rolling, but at the cost of allowing the GM to add a
complication to the scene. This is called "narrative fiat." \index{narrative fiat}

Each player can only use narrative fiat once per scene.

\subsection{Skill Rolls}

Many actions in Vox Draconis are covered by simple attribute
rolls. However, some actions are covered by skills instead.
If an action falls under the area described by a skill\index{skill}, a
character attempting that action must do a skill roll.

A skill roll is an attribute roll, but may also include a "skill
proficiency" modifier. This adds +1 to the skill roll. \index{skill proficiency}
If you would get a specific skill proficiency from more than one source, you gain a bonus of +2 instead of +1.
This is called "skill expertise." You can never gain a +3 bonus, however.\index{skill expertise}

Sometimes, a skill proficiency is required to even attempt a
particular action. For example, a character would need a skill
proficiency in metallurgy to produce an alloy.

Skill proficiencies are granted by classes in character creation.
They may also be learned through the course of gameplay.
See the Character Advancement chapter for further information.

\section{Crisis Rolls}

If a player ever gets into a situation that would have dire
consequences for them, they may make a Crisis Roll\index{crisis roll} to avoid it.
They roll a twenty-sided die, and if the roll is over 10, they
succeed. If they fail, they must spend a Crisis Point to avoid
their fate. If they don't have any Crisis Points to spend, They
suffer the consequences they were trying to avoid.

A Crisis Roll may be used for such things as avoiding death or
mitigating the results of a social engagement.

Crisis Rolls cannot be used to negate a complication.

\section{The Passage of Time}

In combat, a turn lasts roughly 5 seconds of in-game time. Outside
of combat, a turn takes as much or as little in-game time as the
scene demands.\index{turn length}

\section{Travel} \index{travel}

A travel scene is the process of moving from one hex on the world map
to another. Every time the party would move from one hex to another,
they must do a travel scene.

The GM describes the terrain, the weather, and any other relevant
details about the transition. The time it takes to travel is listed
in the table below. Travel along roads will always be quickest. Terrain
with uneven footing is considered "difficult" terrain. Terrain with many
obstacles that must be avoided is considered "very difficult" terrain.
Some terrain, such as sheer mountains, may be completely impassable.

\bottomcaption{Travel times}
\tablefirsthead{\hline \multicolumn{1}{|c|}{\textbf{Terrain Type}} &
                       \multicolumn{1}{c|}{\textbf{On Foot}} &
                       \multicolumn{1}{c|}{\textbf{Mounted}} \\ \hline }
\begin{center}
{\rowcolors{3}{white}{light-gray}
\begin{xtabular}{|l|l|l|}
Road & 1 hour & 30 minutes \\
Normal & 2 hours & 1 hour \\
Difficult & 3 hours & 2 hours \\
Very Difficult & 4 hours & 3 hours \\
\hline
\end{xtabular}
}
\end{center}

\subsection{Travel Events}

Each travel scene, the GM rolls 1d6. If the result is a 1, then the
party encounters a travel event. Roll 1d10 on the travel event table below
to determine its nature.

\bottomcaption{Travel events}
\tablefirsthead{\hline \multicolumn{1}{|c|}{\textbf{Roll}} &
                       \multicolumn{1}{c|}{\textbf{Event}} \\ \hline }
\begin{center}
{\rowcolors{3}{white}{light-gray}
\begin{xtabular}{|l|l|}
1-3 & Bad weather \\
4-6 & Other travellers \\
7-8 & Hostile animals \\
9-10 & Hostile characters \\
\hline
\end{xtabular}
}
\end{center}

\subsubsection{Bad Weather}

Bad weather can take the form of a sandstorm, blizzard, monsoon, or any other
dangerous weather event. The GM describes the approaching weather and gives the
party a chance to prepare for it. The results depend on the severity of the
weather and how well the party prepares. Use imagination and careful description
to determine this.

If the party decides not to take shelter and instead travel through the weather
event, travel time is quadrupled, and each character must make a Physical Toughness
roll of difficulty Hard. If they fail, they suffer harm appropriate to the type
of weather.

\subsubsection{Other Travellers}

A group of other travellers can be either positive, negative, or just uneventful.
This could be something like a passing caravan, a wagon at the side of the road,
a traveling minstrel, or anything else that fits the region. The GM should describe
the travellers and give the party a chance to interact with them. If neither party
chooses to interact, then the travel scene passes uneventfully.

\subsubsection{Hostile Animals}

This type of travel event is a random encounter with a group of hostile animals. The
Allies and Enemies chapter has tables to roll on for each biome for random encounters.
Alternatively, the GM can choose to use a premade encounter of their own.

\subsubsection{Hostile Characters}

Hostile characters are other travellers or denizens of the region that are actively
trying to harm the party. The GM can describe the hostile characters and give the
party a chance to avoid the encounter, or they can just immediately initiate combat.
As with hostile animals, the Allies and Enemies chapter has tables to roll on for
these random encounters.

\subsection{Exploration}

After a travel scene, the party arrives in their destination hex. If there are any
undiscovered hexes neighboring the destination hex, the GM draws them out on the
players' map and describes what they look like from a distance.

Once in a hex, the party may explore it if they have not done so already. Each player
rolls on the following table.

\bottomcaption{Exploration}
\tablefirsthead{\hline \multicolumn{1}{|c|}{\textbf{Roll}} &
                       \multicolumn{1}{c|}{\textbf{Result}} \\ \hline }
\begin{center}
{\rowcolors{3}{white}{light-gray}
\begin{xtabular}{|l|l|}
1-4 & Nothing \\
5-7 & Natural resource \\
8-9 & Point of interest \\
10 & Settlement \\
\hline
\end{xtabular}
}
\end{center}

If the result is natural resource, point of interest, or settlement, that player
rolls on the appropriate table below. The GM makes a note of the result on the
players' map and the GM's map.

\bottomcaption{Natural resources}
\tablefirsthead{\hline \multicolumn{1}{|c|}{\textbf{Roll}} &
                       \multicolumn{1}{c|}{\textbf{Result}} \\ \hline }
\begin{center}
{\rowcolors{3}{white}{light-gray}
\begin{xtabular}{|l|l|}
1 & Metal ore \\
2 & Gem ore \\
3-5 & Plentiful food \\
6-8 & Clean water \\
9-10 & Arable land \\
\hline
\end{xtabular}
}
\end{center}

\bottomcaption{Points of interest}
\tablefirsthead{\hline \multicolumn{1}{|c|}{\textbf{Roll}} &
                       \multicolumn{1}{c|}{\textbf{Result}} \\ \hline }
\begin{center}
{\rowcolors{3}{white}{light-gray}
\begin{xtabular}{|l|l|}
1 & Abandoned fortress \\
2 & Abandoned village \\
3 & Abandoned mine \\
4 & Abandoned temple \\
5 & Abandoned tower \\
6 & Giant plantlife \\
7 & Unusual rock formation \\
8 & Cavern entrance \\
\hline
\end{xtabular}
}
\end{center}

\bottomcaption{Settlements}
\tablefirsthead{\hline \multicolumn{1}{|c|}{\textbf{Roll}} &
                       \multicolumn{1}{c|}{\textbf{Result}} \\ \hline }
\begin{center}
{\rowcolors{3}{white}{light-gray}
\begin{xtabular}{|l|l|}
1-4 & Hamlet \\
5-7 & Village \\
8-9 & Town \\
10 & City \\
\hline
\end{xtabular}
}
\end{center}

\section{Combat}

The following rules govern how combat works.

\subsection{Combat Turns}

On a character's turn in combat, they can take one major action and one minor action.
They can also opt to take two minor actions and forego the major action.

\subsection{Attacking and Defending}

When a character attacks, they roll a twenty-sided die and add their Physical
Might. If the result is higher than the defender's Defense score, the attack hits.\index{attacking}

$$1d20 + PD$$

A character's Defense score is equal to 10 plus the sum of Armor Points they
have. If they're not wearing armor, add their Physical Might.\index{defense score}

$$10 + Body AP + Helmet AP + Shield AP$$

or, for unarmored characters:

$$10 + PD$$

\subsection{Cover}

If the target of an attack is behind cover, the chance to hit them is reduced.
An attacker's attack roll is reduced by the cover penalty listed in the table
below.\index{cover}

\bottomcaption{Cover penalties}
\tablefirsthead{\hline \multicolumn{1}{|c|}{\textbf{Amount of Cover}} &
                       \multicolumn{1}{c|}{\textbf{Penalty}} \\ \hline }
\begin{center}
{\rowcolors{3}{white}{light-gray}
\begin{xtabular}{|l|l|}
Fully hidden & -5 \\
Half hidden & -3 \\
Partly hidden & -1 \\
\hline
\end{xtabular}
}
\end{center}

\subsection{Damage}

A successful hit deals damage equal to the amount by which the attack succeeded.\index{damage}

If the damage a character suffers is greater than half their maximum life points, they receive
a grievous wound. The nature of this wound is up to the GM's discretion.\index{grievous wound}
Grievous wounds can be healed, but they leave a permanent scar. Grievous wounds that impact a
character's abilities - such as severed limbs - should allow the player to make a Crisis Roll
to avoid the grievous wound. If they are successful, they still suffer the life point damage,
but do not take a grievous wound.

\section{Death}

If a character's Life Points reach zero or less, they fall unconscious. When this happens,
the character must make a Crisis Roll. If they fail, they must spend a Crisis Point to stay
alive. If the roll is successful or they spend the Crisis Point, they don't need to continue
rolling, and the character stabilizes but remains unconscious.\index{death}

\section{Healing and Resting}

Characters must rest to recover from wounds. When the party rests for at least 4 hours, each
character regains 1d4 Life Points plus their Physical Toughness.\index{healing}

If the party instead rests for an uninterrupted 8 hours, each character heals up to their maximum
life points.

Note that while life points may return to maximum, grievous wounds may leave scars, and missing
limbs will not grow back.

\section{Ammunition and Limited Resources}

Rather than tracking individual counts of ammunition, rations, and other limited resources, any
player character with such resources will instead use a Resource Die for that resource.\index{resource dice}

A Resource Die reflects the amount of that resource the character has. At full capacity, the
Resource Die is a d8. After any scene where the character uses that resource, the player rolls
the Resource Die. If the result is 1-2, then the Resource Die is reduced by one die type. If
the Resource Die is a d4 and the player rolls a 1-2, then that resource is depleted, and the character
must replenish their stocks before they can use that resource again.

Rations can be replenished by hunting, gathering, or buying food from a settlement. Ammunition
can be replenished at an appropriate store, through crafting, or perhaps by scavenging in a dungeon. The GM's discretion
determines the availability of replacements. Note: skill proficiencies can be helpful here.

\section{Magic}

Magic is based on the manipulation of anima, the mystical energy that suffuses the world of Yrda.
Spellcasting cultures vary in how they use anima and what they use it for. Generally, spellcasters
learn to attune themselves to a particular kind of anima, and their spells reflect that attunement.
Other types of anima become more difficult for them to manipulate, and they may even be completely
unable to use anima that is diametrically opposed to their attunement.

\subsection{Spellcasting}

Casting a spell is a major action. It always uses Anima Might for the roll. If the roll is successful,
the spell is cast. If the roll is more than 5 higher than the target number, the spell is cast with
irresistible force, and it cannot be canceled or dispelled. If the roll is less than the target number,
the spell goes awry, and the anima has an effect that the caster did not intend. The player determines
this effect. If the roll is at least 5 less than the target number, the spell goes badly awry, and the GM
determines the effect.

\subsection{Spells}

There is no set list of spells. Instead, when casting a spell, the player describes what they want the
effect to be. The GM then determines the target number based on the difficulty of the effect.

The character's spells must be related to their anima attunement. If a character is attuned to fire,
they can cast spells that manipulate fire, but they cannot cast spells that manipulate water.

If a spell would cause damage to a living target, the damage dealt is equal to the amount by which the
spell's roll differs from the target number.\index{spell damage}

If the spell is not cast with irresistible force, the target can make an Anima Toughness roll to resist
the spell's damage.\index{spell defense} If the target's roll is higher than the spell's roll, they avoid
any damage from the spell. Effects other than damage still occur.

Spells that have ongoing effects which target a living creature can be broken on a future turn, either by
the caster at will or by the target. For the target to do it, they must make an Anima Toughness roll of
difficulty Moderate. If they succeed, the spell is broken.\index{spell breaking}

\section{Social Engagements}

A social engagement is a conversation of importance between two or more sides. Before it begins,
divide the participants into sides. Every participant will argue for their side. Each side
determines what they're arguing for before the engagement begins.

Each participant has a turn. On their turn, they may choose to take one of the
following actions, or give up their turn for the round. Every social engagement
lasts for three rounds. At the end of the three rounds, the total points of each
side is compared, and the highest wins. In the event of a tie, there is no
victor, and the engagement is a draw.\index{social engagement}

No side can go below zero points.

Once the engagement is over, the players as a group determine the result of the winning argument.

\subsection{Actions in Social Engagements}

There are three types of actions a participant can take in a social engagement.
These are Persuade, Dissuade, or Riposte. The target number for Persuade and
Dissuade is set by the GM based on how difficult the point they're trying to
make is. Use the target number table from the Actions section.

\subsubsection{Persuade}

This is an attempt to convince another participant of the correctness of your argument. Make a
Social Might attribute roll. The target number is a difficulty the GM sets modified by the other
participant's Social Toughness. If you succeed, you gain one point for your side.

$$1d20 + SS$$

\subsubsection{Dissuade}

You can try and reduce the points of another participant's side. Roll a twenty-sided die
and add your Social Might. The target number is modified by the other participant's
Social Might. If you win, the other side loses a point.

$$1d20 + SD$$

\subsubsection{Riposte}

Instead of taking a direct action, you can complicate the result of the entire engagement
by making a Riposte. Roll a twenty-sided die and add your Social Might. If the result
is higher than 10, then regardless of the result of the social engagement, there will be
a complication. The GM chooses the complication.

$$1d20 + SD$$

\section{Experience Points}

This is the reward that allows player characters to grow in ability. The Game Master gives
out Experience Points at the end of a session based on the players' actions during the
session. Experience Points can be awarded for any action by the player characters that
has a noticeable impact on the game world. See the Character Advancement chapter for
more details.\index{experience points}

\section{Player Points}

While characters can gain in power as a reward for their active participation in the game
world, players should also gain a reward for improving the game experience for everyone.
The reward for this is Player Points. Any player at the table, not just the Game Master,
can give these out. However, only one can be received by each player per session.

Player Points can be spent to alter or enhance the player character's backstory, appearance,
or other aspects of the character outside of their game mechanics. The Game Master should
use their discretion in determining the cost of such changes.\index{player points}

See the Character Advancement chapter for more details.

\end{multicols}
