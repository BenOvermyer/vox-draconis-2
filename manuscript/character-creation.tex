\chapter{Character Creation}

\begin{multicols}{2}

\section{Creating a Character}

These rules determine how players can create their own characters.

\section{Attributes}

Every character has the following attributes. These define their 
physical, mental, social, and magical capabilities. Each of these 
attributes has a positive or negative score. They range from -3 to +3. 
These scores modify rolls based on those attributes.

There are four categories: Physical, Mental, Social, and Arcane. 
Each category has the following attributes.

\begin{itemize}
    \item \textbf{Strength:} Offensive ability or power.
    \item \textbf{Endurance:} Defensive ability or resistance.
    \item \textbf{Dexterity:} Manipulative ability or agility.
\end{itemize}

In total, there are twelve attributes. For brevity's sake, sometimes 
you will see references to abbreviations of these attributes. These 
abbreviations will always have the first letter of the category and 
the first letter of the attribute. So, Physical Strength would be 
abbreviated PS, and Arcane Endurance would be abbreviated AE.

\subsection{Determine Initial Attributes}

At the beginning of character creation, assign the following scores to 
your attributes.

\begin{itemize}
    \item 2 scores of +2
    \item 3 scores of +1
    \item 4 scores of 0
    \item 2 scores of -1
    \item 1 score of -2
\end{itemize}

Note that the best and worst scores (+3 and -3, respectively) are not 
included in character creation. As game events alter characters, 
attributes may increase or decrease. They will never go below -3 or
above +3, however.

\section{Choose Two Ancestries}

Pick two ancestries. This will give you several innate abilities 
and a piece of your character's background. Ancestries are detailed
in the Ancestries chapter.

\section{Choose a Class}

Pick a class. This will determine your path in life and which 
abilities you have refined. Classes are detailed in the Classes
chapter.

\section{Choose an Attunement}

Pick an attunement. This will grant you a supernatural ability 
and an affinity for a supernatural force. Be warned: that 
affinity will be both a boon and a weakness. Attunements are
detailed in the Attunements chapter.

\section{Derived Statistics}

After all the above is accounted for, calculate your derived 
statistics. These are Life Points and Crisis Points. Life Points 
determine how much damage your character can take before falling 
unconscious. Crisis Points determine how many times your character 
can narrowly avoid disaster. They are calculated as follows.

\begin{itemize}
    \item \textbf{Life Points:} 10 + Physical Endurance
    \item \textbf{Crisis Points:} Mental Endurance + Physical Endurance + Social Endurance + Arcane Endurance
\end{itemize}

\section{Starting Equipment}

Your character begins play with 50 silver drachms, the coinage of 
the realm. You must spend at least half of it on equipment from the 
Equipment chapter.

\section{Languages}

All player characters understand how to speak Common, the language most
widely spoken in Yrda. This comes as the universal skill proficiency
\textit{Language (Common) +1}. \index{Common language} \index{language}

In lore terms, "Common" is also known by its older name, Yrdish.

Sometimes you will have the opportunity to gain new languages. All languages
are represented by two skill proficiency types: \textit{Language} and \textit{Literacy}.
Language lets you speak a language, and Literacy lets you read it.

The following are some of the languages in Yrda:

\begin{itemize}
    \item Yrdish (or Common) - the common language of Yrda
    \item Old Yrdish - a more archaic version of Yrdish, this is no longer spoken
    outside of scholarly circles or some lost cultures
    \item Makhetian - the language of the Makhet culture, spoken primarily in the 
    Kingdom of Makhet
    \item Ardonan - the language of the Ardonan culture, spoken primarily in the 
    Kingdom of Ardona and western parts of the Sushani Empire
    \item Sushani - the language of the Sushani culture, spoken primarily in the 
    Sushani Empire
\end{itemize}

\end{multicols}
