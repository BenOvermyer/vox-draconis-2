\chapter{Character Creation}

\begin{multicols}{2}

\section{Creating a Character}

These rules determine how players can create their own characters.

\section{Attributes}

Every character has the following attributes. These define their
physical, mental, social, and magical capabilities. Each of these
attributes has a positive or negative score. They range from -3 to +3.
These scores modify rolls based on those attributes.

There are four categories: Physical, Mental, Social, and Anima.
Each category has the following attributes.

\begin{itemize}
  \item \textbf{Might:} Offensive ability or power.
  \item \textbf{Toughness:} Defensive ability or resistance.
\end{itemize}

Physical attributes govern your character's ability to physically
interact with the world.

Mental attributes govern your character's ability to think, reason,
and plan.

Social attributes govern your character's ability to interact with
others.

Anima attributes govern your character's ability to manipulate
anima, the magical energy that flows through the world.

In total, there are eight attributes. For brevity's sake, sometimes
you will see references to abbreviations of these attributes. These
abbreviations will always have the first letter of the category and
the first letter of the attribute. So, Physical Might would be
abbreviated PM, and Anima Toughness would be abbreviated AT.

\subsection{Determine Initial Attributes}

At the beginning of character creation, assign the following scores to
your attributes.

\begin{itemize}
  \item 2 scores of +2
  \item 2 scores of +1
  \item 3 scores of 0
  \item 1 score of -1
\end{itemize}

Note that the best and worst scores (+3 and -3, respectively) are not
included in character creation. As game events alter characters,
attributes may increase or decrease. They will never go below -3 or
above +3, however.

\section{Choose an Ancestry}

Choose your ancestry. This will give you several innate abilities
and a piece of your character's background. Ancestries are detailed
in the Ancestries chapter.

\section{Choose a Class}

Pick a class. This will determine your path in life and which
abilities you have refined. Classes are detailed in the Classes
chapter.

\section{Derived Statistics}

After all the above is accounted for, calculate your derived
statistics. These are Life Points, Crisis Points, and Initiative. Life Points
determine how much damage your character can take before falling
unconscious. Crisis Points determine how many times your character
can narrowly avoid disaster. Initiative determines turn order in combat.
They are calculated as follows.

\begin{itemize}
  \item \textbf{Life Points:} 10 + Physical Toughness
  \item \textbf{Crisis Points:} Mental Toughness + Physical Toughness + Social Toughness + Anima Toughness
  \item \textbf{Initiative:} 5 + Mental Might + Physical Might
\end{itemize}

\section{Starting Equipment}

In addition to the starting equipment given to you by your ancestry and class,
you also gain 2d6 x 10 copper coins.

\section{Starting Language}

All player characters understand how to speak Yrdish, the language most
widely spoken in Yrda. This comes as the skill proficiency
"Language (Yrdish)". \index{language, Yrdish} \index{language}

Note: language knowledge is split into two skill proficiencies. The "Language"
proficiency is the ability to speak and understand a language. The "Literacy"
proficiency is the ability to read and write a language. You only start with
the "Language (Yrdish)" proficiency and not the "Literacy (Yrdish)" proficiency.

At character creation, you may choose one additional language to be fluent in,
or you may take the "Literacy (Yrdish)" proficiency instead. See the "languages"
section of the World of Yrda chapter for more information on available languages.

For convenience, here is a short list of additional languages available:

\begin{itemize}
  \item Ardonan
  \item Gaddari
  \item Makhetian
  \item Old Yrdish
  \item Sushani
\end{itemize}

\end{multicols}
