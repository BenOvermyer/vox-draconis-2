\chapter{Classes}

\begin{multicols}{2}

\section{About Classes}

The following are the classes of Vox Draconis. They are professions,
followings, or other life pursuits. Each class will give you some
background information, some unique abilities, and a set of skill
proficiencies.

\section{List of Classes}

\subsection{Dragon-Bonded}

If a child is present when a dragon's egg hatches, there is a small
chance that the hatchling and the child will form a supernatural and
unbreakable bond. The Illdrazi have built their entire culture around
this phenomenon, and co-exist in peace with dragons.

A dragon-bonded's life is built around that bond, for better or worse.
They live with, hunt with, grow old with, and die with their dragon.

At character creation, choose one of the dragon types from the
Enemies and Allies chapter. You gain a dragon bond-mate of that
type. They will listen to you and usually do as asked, but are
not mindless servants and have goals of their own. Consider the
personality of your bond-mate.

\subsubsection{Dragon-Bonded Abilities}

\begin{itemize}
  \item \textbf{Shared Life:} You are bonded to a single dragon for
    life. If your bond-mate dies, you must make a Crisis Roll. If you
    fail and cannot spend a Crisis Point, you also die. However, you
    also gain the lifespan of your dragon, and will not die of old age until
    they do. If you die, your dragon dies.
  \item \textbf{Telepathic Link:} You can communicate with your bond-mate in
    words and images, no matter how much distance is between you.
\end{itemize}

\subsubsection{Dragon-Bonded Skill Proficiencies}

\begin{itemize}
  \item Dragonriding
  \item Dragon Medicine
  \item Lance
\end{itemize}

\subsubsection{Dragon-Bonded Equipment}

\begin{itemize}
  \item Roll 1d6 on random weapons table
  \item Roll 1d6 on random armor table
  \item A dragon-saddle
\end{itemize}

\subsection{Raptor-Bonded}

The Ardonans are fierce tribal warriors whose lives revolve around
fighting. They live in the jungles of Ardona, home to many equally
fierce beasts. Among these are the greater raptors - bipedal carnivorous
dinosaurs the size of a horse. When a clutch of greater raptor eggs hatch,
a tribe of Ardonans will sometimes steal the hatchlings and raise them
alongside their young.

Often, a raptor hatchling and a child will form a supernatural bond.
Once this occurs, the bond lasts until one of them dies.

The raptor-bonded and their bond-mate hunt, fight, and live together.
Groups of Ardonan raptor-bonded will hunt as packs and are deeply loyal
to each other.

\subsubsection{Raptor-Bonded Abilities}

\begin{itemize}
  \item \textbf{Shared Senses:} If you concentrate, you can share all
    of the senses of your bond-mate. You can only do this for a few seconds
    before needing to break the link, and you can only do it once every hour.
    The senses of your bond-mate replace your own for the duration.
  \item \textbf{Empathic Link:} You can communicate with your bond-mate
    in emotions and images, no matter how much distance is between you.
\end{itemize}

\subsubsection{Raptor-Bonded Skill Proficiencies}

\begin{itemize}
  \item Raptor-riding
  \item Raptor Medicine
  \item Hunting
  \item Spear
\end{itemize}

\subsubsection{Raptor-Bonded Equipment}

\begin{itemize}
  \item A raptor-saddle
  \item Roll 1d6 on random weapons table
  \item Roll 1d6 on random armor table
\end{itemize}

\subsection{Knight of the Realm}

Knights of the Realm are trained warriors who adhere to a strict code of
honor. They hail from many different kingdoms, and owe allegiance to a
noble family. When a member of that noble family calls for their service,
they must give it. A Knight of the Realm is almost always a noble themselves.

Knights are as skilled in etiquette and courtly ways as they are in combat.

\subsubsection{Knight of the Realm Abilities}

\begin{itemize}
  \item \textbf{Shield Another:} Once per scene, you may take a blow
    intended for another. You must be within range to interject yourself.
  \item \textbf{Bond of Honor:} You may make a Bond of Honor with another
    person by jointly speaking a ritual phrase beginning with "I bind myself
    to..." and ending with "...and to this be true." So long as at least one
    participant is alive, the bond requires both to adhere to the words of
    the Bond. Breaking the Bond results in catastrophic bad luck for the
    one who breaks it. Only one Bond may be active at a time. If the Bond's
    wording allows for it to be completed, completion of the Bond ends it
    without ill effect.
\end{itemize}

\subsubsection{Knight of the Realm Skill Proficiencies}

\begin{itemize}
  \item Horsemanship
  \item Etiquette
  \item Long Sword
  \item Heraldry
\end{itemize}

\subsubsection{Knight of the Realm Equipment}

\begin{itemize}
  \item Roll 1d10 on random weapons table
  \item Roll 1d10 on random armor table
  \item A signet ring with the crest of your patron noble house
\end{itemize}

\subsection{Adjudicator}

The kingdom of Makhet has a strong tradition of law and order. The
Adjudicators are independent interpreters of the law, and by necessity,
they are also highly trained warriors. Their role is to see to it that
justice is served and that lawbreakers do not go unpunished. They are
the king's instruments, and have his full authority behind them. As
such, an Adjudicator's word has the force of law.

Some Adjudicators are restricted to a single city, but most travel from
town to town. Any citizen can call upon them to hear a report of law
breaking, and once an Adjudicator answers such a call, the matter is
fully within their hands to resolve.

Outside of Makhet, they have no authority, though some of them seem to
labor under the illusion that they do.

\subsubsection{Adjudicator Abilities}

\begin{itemize}
  \item \textbf{Discern Truth:} An Adjudicator, as part of their final
    investment of office, undergoes a ritual which permanently allows them
    to tell when someone is telling the truth or not. They can't tell
    what the truth \textit{is}, only that the speaker believes what
    they're saying.
  \item \textbf{Oath of Makhet:} The investment ritual also bonds the
    Adjudicator permanently to the monarchy of Makhet. The monarch
    possesses a small gem that, if shattered, will kill the Adjudicator
    instantly.
\end{itemize}

\subsubsection{Adjudicator Skill Proficiencies}

\begin{itemize}
  \item Detect Forgery
  \item Interrogation
  \item Intimidation
  \item Literacy (choose two languages)
  \item Persuasion
  \item Short Sword
\end{itemize}

\subsubsection{Adjudicator Equipment}

\begin{itemize}
  \item Roll 1d4 on the random weapons table
  \item Roll 1d6 on the random armor table
  \item A book of the law
\end{itemize}

\subsection{Fell Knight}

The followers of the Fell Titan all come from different backgrounds.
They share one thing in common: a desperate need for power. Each Fell
Knight made the pact with the Fell Titan under different circumstances
and for different reasons, but now they all serve the Titan's will.

These deadly warriors are forever changed by the pact. In exchange for
their unquestioning service, they are given frightening powers. The longer
a Fell Knight has been in service, the more they physically change. Over
time, their features become more drawn, worn, and emaciated. Eventually,
after several years, their flesh - if they have it - disappears entirely,
and they become skeletal horrors. Unlike mere undead, however, they retain
their full intellect and free will, except where such will conflicts with
the direction of the Fell Titan.

\subsubsection{Fell Knight Abilities}

\begin{itemize}
  \item \textbf{Wasting Immortality:} A Fell Knight cannot die of old age,
    instead becoming an animated skeleton over time.
  \item \textbf{Drain Life:} The touch of a Fell Knight can drain the life
    force of any living thing. While this causes no overt damage, it reduces
    the lifespan of the creature or plant in question, aging them visibly.
    This ability can only be used once per week, as it weakens the Fell
    Knight's connection to the Titan. The reduction in lifespan is
    approximately 20\% of their current maximum.
\end{itemize}

\subsubsection{Fell Knight Skill Proficiencies}

\begin{itemize}
  \item Battle Axe
  \item Mace
  \item Intimidation
\end{itemize}

\subsubsection{Fell Knight Equipment}

\begin{itemize}
  \item Roll 1d8 on the random weapons table
  \item Roll 1d6 on the random armor table
  \item A black cloak
\end{itemize}

\subsection{Star Sage}

Star sages are scholars and philosophers with deep knowledge of the heavens.
Some groups of them have built massive observatories hidden in mountains. They
train others to see and interpret the movements of heavenly bodies. Some people
think they are little more than charlatans. Others have seen the strange powers
that such a knowledge has given them.

Star sages often spend time at observatories, studying the heavens. Some of their
number, however, spend just as much time wandering Yrda, applying the skills they
have acquired. They act as fortunetellers, yes, but also as engineers and architects.
Star sages' skill at building marvelous machines is unquestioned.

\subsubsection{Star Sage Abilities}

\begin{itemize}
  \item \textbf{Wonderful Toys:} Star sages can build contraptions of gears, belts,
    steam, and other strange and advanced technology. Only the star sage that built
    the device can operate it, as it has quirks that only they know about. The larger
    the device, and the more complex its operation, the longer it takes to build.
  \item \textbf{Read the Stars:} By reading the heavens and consulting their charts,
    star sages can predict the future. This takes at least an hour, and must be done
    at night, with a full view of the stars. These predictions are vague; rather than
    "Brutus will stab you in the back at 3:00 PM this afternoon," a star sage's prediction
    would be "a man you know and trust will bring harm to you today."
\end{itemize}

\subsubsection{Star Sage Skill Proficiencies}

\begin{itemize}
  \item Engineering
  \item Astronomy
  \item Astrology
  \item Mathematics
  \item Literacy (Common)
  \item Language (one additional language)
  \item Literacy (one additional language)
\end{itemize}

\subsubsection{Star Sage Equipment}

\begin{itemize}
  \item Roll 1d4 on the random weapons table
  \item A telescope
  \item A sextant
  \item A pocket watch
  \item A quill pen
  \item A book of charts
\end{itemize}

\subsection{Deathstalker}

Yrda is no stranger to battle. Many have died in combat. The Deathstalker
is a warrior who has come to not only accept death in combat as a possibility,
but to embrace it as the one true path. Deathstalkers worship death. The
battlefield is their temple, and violence is their ritual.

Most deathstalkers operate as mercenaries. Their reputation for wholesale
slaughter and religious disregard for mercy and restraint means that they
are both feared and command a high price for their services.

Deathstalkers see it as their holy duty to kill. They restrain themselves
from random murder as a necessary evil; by adhering to mortal laws, they
are able to end more lives over time. However, when law does not restrain
them, they are forces of unmitigated destruction.

Curiously, their affinity for death has also granted them some skill in
delaying it. There have been more than a few practitioners of medicine
who discover that their natural talent for healing came about as a side
effect of their true calling, and have left their old profession to become
a deathstalker.

\subsubsection{Deathstalker Abilities}

\begin{itemize}
  \item \textbf{Sense Death:} A Deathstalker can sense death within a
    mile radius. They know how recently it occurred and how it occurred.
  \item \textbf{Visage of Death:} Deathstalkers have a gaunt appearance.
    Their skin is pale, their eyes are sunken and dark, and they perpetually
    smell faintly of grave dirt.
\end{itemize}

\subsubsection{Deathstalker Skill Proficiencies}

\begin{itemize}
  \item Scythe
  \item Short Sword
  \item Long Sword
  \item Intimidation
  \item Medicine
\end{itemize}

\subsubsection{Deathstalker Equipment}

\begin{itemize}
  \item Roll 1d10 on the random weapons table
  \item Roll 1d4 on the random armor table
  \item A steel neck chain with an onyx skull amulet
\end{itemize}

\subsection{Swashbuckler}

Swashbucklers are flamboyant adventurers with a penchant for the dramatic.
They adhere to a code of honor, protecting the downtrodden and duelling
villainous scoundrels. They show a fondness for ostentatious clothing, and
often come from noble backgrounds.

Despite their apparent bravado, swashbucklers are skilled at swordplay and
display a cunning not limited to battle.

\subsubsection{Swashbuckler Abilities}

\begin{itemize}
  \item \textbf{Undeniable Duel:} The swashbuckler may declare a duel between
    themselves and a villainous character. The target must make a Hard Mental
    Endurance roll; if they fail, they must fight the swashbuckler and only the
    swashbuckler until one of them is rendered unconscious or dead.
  \item \textbf{Daring Acrobatics:} Once per scene, the swashbuckler may
    dramatically change their location through incredible agility without having
    to make a Physical Dexterity roll. This includes things like leaping from a
    rooftop to the ground, cutting the rope holding a chandelier and riding it to
    a second floor, and so on. It must be dramatic and there must be an audience.
\end{itemize}

\subsubsection{Swashbuckler Skill Proficiencies}

\begin{itemize}
  \item Rapier
  \item Cutlass
  \item Charm
  \item Acrobatics
  \item Dance
\end{itemize}

\subsubsection{Swashbuckler Equipment}

\begin{itemize}
  \item Roll 1d4 on the random weapons table
  \item Roll 1d4 on the random armor table
  \item A rapier OR a cutlass
  \item A set of fancy clothes
\end{itemize}

\subsection{Spirit Talker}

People of the Inyani tribes have long used the flowering herb \textit{jahrah} \index{jahrah}
to weaken the wall between the worlds of the living and the dead. Each tribe
has at least one person who acts as a representative to the spirits. These
"Spirit Talkers" have occasionally taught their arts to those outside of the
Inyani.

Outside of the Inyani tribes, spirit talkers take many forms. They are
particularly valued in the Kingdom of Makhet for discerning the truth
behind murders.

\subsubsection{Spirit Talker Abilities}

\begin{itemize}
  \item \textbf{Spirit Trance:} A spirit talker smokes the herb jahrah and
    enters a trance-like state where they can see and talk to the spirits of
    the dead. This trance lasts for about half an hour per puff of jahrah. The
    spirits they interact with are either recently dead or are haunting the
    local area.
  \item \textbf{Altered Mind:} Prolonged use of jahrah has altered the spirit
    talker's mind permanently. They can sense when spirits are nearby, and are
    able to sense the emotions of those spirits, even when not in a trance.
\end{itemize}

\subsubsection{Spirit Talker Skill Proficiencies}

\begin{itemize}
  \item Persuasion
  \item Calm
  \item Perception
  \item Gardening
\end{itemize}

\subsubsection{Spirit Talker Equipment}

\begin{itemize}
  \item Roll 1d4 on random weapons table
  \item Roll 1d4 on random armor table
  \item A pouch of jahrah
  \item A pipe
\end{itemize}

\subsection{Soul Hunter}

Soul hunters are highly trained warriors whose entire lives revolve around
finding unnatural corruption of souls and destroying it. No one knows where
they first originated, and they are not numerous. Their abilities stem from
a natural talent for sensing supernatural corruption. Once a soul hunter \index{supernatural corruption}
finds a person with this talent, it is their sworn duty to train that person
as a soul hunter.

Supernatural corruption is a tainting of individuals by the Unknowable Void, \index{Unknowable Void}
that vast and powerful force that can consume even gods. If it grows too strong
in any area, Void-tainted creatures begin to spring up, madly attacking anything
and everything around them. These creatures are so dangerous that it takes
multiple soul hunters to destroy even a single one. As such, it is vital that
supernatural corruption is found and destroyed before the Void-tainted can appear.

Once a year, all soul hunters return to the Den of Purity, an ancient and massive
structure somewhere in the Wild Tooth Mountains. There, they train and
participate in rituals to cleanse their own souls of any corruption that might
be lurking there.

\subsubsection{Soul Hunter Abilities}

\begin{itemize}
  \item \textbf{Cleanse Soul:} The soul hunter plants his weapon blade-first into
    the ground and stretches a hand out towards a being with a corrupted soul. For
    the next two rounds, the soul hunter must roll Arcane Dexterity against a difficulty
    the GM determines. If both rolls succeed, all supernatural corruption is drawn
    out of the being and dispersed. If the being is a Void-tainted creature, this
    difficulty is always Very Hard. A Void-tainted creature returns to normal if
    this process succeeds.
  \item \textbf{Clarity of Mind:} Soul hunters train for hours each day to maintain
    their skill and hone their abilities. As a result of this training, they have
    the ability to clear their minds completely, instantly calming themselves and rendering
    them temporarily immune to psionic or empathic attacks or attempts to read
    their minds. This does not require a roll but can only be sustained for two rounds.
\end{itemize}

\subsubsection{Soul Hunter Skill Proficiencies}

\begin{itemize}
  \item Meditation
  \item Short Sword
  \item Trident
  \item Acrobatics
  \item Calm
\end{itemize}

\subsubsection{Soul Hunter Equipment}

\begin{itemize}
  \item Roll 1d4 on the random weapons table
  \item Roll 1d4 on the random armor table
  \item A cloudy crystal
\end{itemize}

\subsection{Cleric of the Light}

The outspoken adherents of the Path of Three who believe that the
light has the most to teach of the three forces are called Clerics
of the Light. They bind themselves to symbols of the light, such as
the sun, moon, stars, fire, and so on. They view the projection of
their will onto the world as a divine calling, and some manifest
powers associated with light.

Clerics of the Light are found in positions of leadership or advisement
all over Yrda where the Path of Three holds sway.

\subsubsection{Cleric of the Light Abilities}

\begin{itemize}
  \item \textbf{Illuminate:} Clerics of the Light can imbue a small object
    with light equivalent to candlelight in brightness at will. Once lit,
    it is only extinguished when the cleric so imbues another object.
  \item \textbf{Sword of the Light:} Some few clerics can call forth a
    blade made of concentrated light out of nothing. This can be done once
    per day, and lasts for a single scene. It always takes the form of the
    first bladed weapon that cleric used in anger, and deals the same damage.
    It is bright to look upon, but not blindingly so.
\end{itemize}

\subsubsection{Cleric of the Light Skill Proficiencies}

\begin{itemize}
  \item Leadership
  \item Intimidation
  \item Persuasion
  \item Short Sword
\end{itemize}

\subsubsection{Cleric of the Light Equipment}

\begin{itemize}
  \item Roll 1d4 on the random weapons table
  \item Roll 1d4 on the random armor table
  \item A holy symbol of the light
\end{itemize}

\subsection{Cleric of the Dark}

The darkness teaches some adherents of the Path of Three that shadow and stillness
bring the most benefit to all. Clerics of the Dark practice calm, measured
action and meditation. They bear symbols of darkness, such as black cloaks,
black circles, etc. They view watchfulness and patience as a divine calling.
Some manifest powers associated with darkness.

\subsubsection{Cleric of the Dark Abilities}

\begin{itemize}
  \item \textbf{Cloak in Darkness:} Clerics of the Dark can hide a small object
    from view even in direct sight and full light. Once done, that object can only
    be perceived by them or other clerics of the Dark until it is destroyed or
    another object is so hidden.
  \item \textbf{Vanish:} A rare few clerics can give themselves so fully to the
    Dark that they vanish completely from sight. This only lasts for at most a single
    scene and can only be done once per week. While in this state, they may not
    interact with the world other than to move about it.
\end{itemize}

\subsubsection{Cleric of the Dark Skill Proficiencies}

\begin{itemize}
  \item Dissuasion
  \item Calm
  \item Meditation
  \item Hiding
\end{itemize}

\subsubsection{Cleric of the Dark Equipment}

\begin{itemize}
  \item Roll 1d4 on the random weapons table
  \item Roll 1d4 on the random armor table
  \item A holy symbol of the dark
\end{itemize}

\subsection{Cleric of the Balance}

The third part of the Path of Three is Balance. Clerics who devote themselves to
this path believe that balance in all things has the most to teach the world.
They maintain that balance through any means at their disposal. Some have a more
nuanced view of what "balance" entails than others. Clerics of the Balance wear
symbols associated with balance, such as scales. They view preventing one side
in any given conflict from getting too strong as a divine calling. Some manifest
powers associated with balance.

\subsubsection{Cleric of the Balance Abilities}

\begin{itemize}
  \item \textbf{Neutralize:} Clerics of the Balance can neutralize the powers of
    Clerics of the Dark and Clerics of the Light at will.
  \item \textbf{Mirror:} Once per day, a cleric of the Balance can copy an ability
    someone else has just used and use it against them.
\end{itemize}

\subsubsection{Cleric of the Balance Skill Proficiencies}

\begin{itemize}
  \item Persuasion
  \item Short Sword
  \item Perception
\end{itemize}

\subsubsection{Cleric of the Balance Equipment}

\begin{itemize}
  \item Roll 1d4 on the random weapons table
  \item Roll 1d4 on the random armor table
  \item A holy symbol of the balance
\end{itemize}

\end{multicols}
