\chapter{Ancestries}

\begin{multicols}{2}

\section{About Ancestries}

Your character's ancestry is an important part of their story. It gives
context for where you came from and how you grew up. It also tells you
what you look like. It does \textit{not} tell you what your future will
be. That is up to you.

There are three parts to a character's ancestry in Vox Draconis. The first
is your species. The second is your culture. The third is your biome.

Your species gives you physical traits and some special abilities. Your
culture gives you some ingrained behaviors and outlooks and a few perks. Your biome gives
you some physical traits and skill proficiencies.

\subsection{Mixed Species}

If you choose to be of mixed descent, choose one species to be your primary
bloodline. Most of your appearance and your special ability will come from this. Then
choose a second species to influence your appearance. The magic-imbued world
of Yrda causes many strange combinations to be viable, so be creative in your
descriptions.

\begin{displayquote}
Alex decides that her species will be Human. She then picks Farming Village
for culture, and Coast for her biome.
\end{displayquote}

\section{List of Species}

Choose one or roll 1d10 on the table below to determine your species randomly.
If you get the "Mixed-species" result, then roll again twice, rerolling duplicates.

\bottomcaption{Random species}
\tablefirsthead{\hline \multicolumn{1}{|c|}{\textbf{Roll}} &
                       \multicolumn{1}{c|}{\textbf{Species}} \\ \hline }
\begin{center}
{\rowcolors{2}{white}{light-gray}
\begin{xtabular}{|l|l|}
1-4 & Human \\
5 & Dwarf \\
6 & Elf \\
7 & Mudling \\
8 & Orc \\
9 & Stone-born \\
10 & Mixed-species \\
\hline
\end{xtabular}
}
\end{center}

\subsection{Dwarf}

Dwarves are hardy beings with stout stature and volumnous beards.\index{dwarves}

\bottomcaption{Height and weight for dwarves}
\tablefirsthead{\hline \multicolumn{1}{|c|}{\textbf{Gender}} &
                       \multicolumn{1}{c|}{\textbf{Height}} &
                       \multicolumn{1}{c|}{\textbf{Weight (lbs.)}} \\ \hline }
\begin{center}
{\rowcolors{3}{white}{light-gray}
\begin{xtabular}{|l|l|l|}
Female & 3 ft. + 2d4 in. & 100 + 2d10 \\
Male & 3 ft. + 2d8 in. & 130 + 2d10 \\
\hline
\end{xtabular}
}
\end{center}

\textbf{Special Ability:} Dwarves are highly resistant to the effects of alcohol and other poisons. Any time they might normally be affected by such a substance, roll 1d10. They are only affected on a roll of a 1.

\subsection{Elf}

Elves are thin and graceful. Their eyes are almond-shaped, and they
have long, pointed ears.\index{elves}

\bottomcaption{Height and weight for elves}
\tablefirsthead{\hline \multicolumn{1}{|c|}{\textbf{Gender}} &
                       \multicolumn{1}{c|}{\textbf{Height}} &
                       \multicolumn{1}{c|}{\textbf{Weight (lbs.)}} \\ \hline }
\begin{center}
{\rowcolors{3}{white}{light-gray}
\begin{xtabular}{|l|l|l|}
Female & 4 ft. + 2d6 in. & 80 + 2d10 \\
Male & 4 ft. + 2d12 in. & 85 + 2d10 \\
\hline
\end{xtabular}
}
\end{center}

\textbf{Special Ability:} Elves do not sleep and cannot be put to sleep, though they can be knocked unconscious. They still require at least four hours of rest per day, however.

\subsection{Human}

Humans are the most common sentient species on Yrda.\index{human}

\bottomcaption{Height and weight for humans}
\tablefirsthead{\hline \multicolumn{1}{|c|}{\textbf{Gender}} &
                       \multicolumn{1}{c|}{\textbf{Height}} &
                       \multicolumn{1}{c|}{\textbf{Weight (lbs.)}} \\ \hline }
\begin{center}
{\rowcolors{3}{white}{light-gray}
\begin{xtabular}{|l|l|l|}
Female & 4 ft. + 2d10 in. & 85 + 2d20 \\
Male & 5 ft. + 1d10 in. & 120 + 4d20 \\
\hline
\end{xtabular}
}
\end{center}

\textbf{Special Ability:} Humans heal quickly. They heal 1 extra life point per 2 hours spent resting.

\subsection{Mudling}

Mudlings are small beings with elongated heads, large eyes, and four fingers on each hand. They
have thin or no hair. Their skin is usually shades of grey, green, or blue.\index{mudling}

\bottomcaption{Height and weight for mudlings}
\tablefirsthead{\hline \multicolumn{1}{|c|}{\textbf{Gender}} &
                       \multicolumn{1}{c|}{\textbf{Height}} &
                       \multicolumn{1}{c|}{\textbf{Weight (lbs.)}} \\ \hline }
\begin{center}
{\rowcolors{3}{white}{light-gray}
\begin{xtabular}{|l|l|l|}
Female & 2 ft. + 1d8 in. & 30 + 2d6 \\
Male & 2 ft. + 2d8 in. & 40 + 2d6 \\
\hline
\end{xtabular}
}
\end{center}

\textbf{Special Ability:} Mudlings can hold their breath for up to half an hour without ill effect.

\subsection{Orc} \index{orcs}

Orcs have green, grey, or earth-brown skin and black or brown hair. They have
elongated faces resembling a cross between a wild boar's and a human's. Orcs
are stronger and larger than humans on average.

\bottomcaption{Height and weight for orcs}
\tablefirsthead{\hline \multicolumn{1}{|c|}{\textbf{Gender}} &
                       \multicolumn{1}{c|}{\textbf{Height}} &
                       \multicolumn{1}{c|}{\textbf{Weight (lbs.)}} \\ \hline }
\begin{center}
{\rowcolors{3}{white}{light-gray}
\begin{xtabular}{|l|l|l|}
Female & 6 ft. + 1d6 in. & 180 + 2d10 \\
Male & 6 ft. + 2d6 in. & 200 + 2d10 \\
\hline
\end{xtabular}
}
\end{center}

\textbf{Special Ability:} Orcs have a powerful sense of smell. They can smell - and track by smell - ten times further and stronger than normal.

\subsection{Stone-Born}

Stone-born are tall beings with tough, rocklike skin and glowing eyes. They are hairless. Some
possess growths of rough gemstone on their heads where humans would have hair or beards. Their
skin can have the color and pattern of any natural stone.\index{stone-born}

\bottomcaption{Height and weight for stone-born}
\tablefirsthead{\hline \multicolumn{1}{|c|}{\textbf{Gender}} &
                       \multicolumn{1}{c|}{\textbf{Height}} &
                       \multicolumn{1}{c|}{\textbf{Weight (lbs.)}} \\ \hline }
\begin{center}
{\rowcolors{3}{white}{light-gray}
\begin{xtabular}{|l|l|l|}
Female & 6 ft. + 2d6 in. & 160 + 2d10 \\
Male & 5 ft. + 1d6 in. & 140 + 2d10 \\
\hline
\end{xtabular}
}
\end{center}

\textbf{Special Ability:} Stone-born are highly resistant to temperature extremes. Any time they might be affected by extreme temperatures, roll 1d6. Only on a 1 are they affected.

\section{List of Cultures}

Cultures describe the community that you grew up in. \index{cultures}

Choose one or roll 1d6 on the following table.

\bottomcaption{Random cultures}
\tablefirsthead{\hline \multicolumn{1}{|c|}{\textbf{Roll}} &
                       \multicolumn{1}{c|}{\textbf{Culture}} \\ \hline }
\begin{center}
{\rowcolors{2}{white}{light-gray}
\begin{xtabular}{|l|l|}
1 & Farming Village \\
2 & Large City \\
3 & Nomad Tribe \\
4 & Religious Order \\
5 & Traveling Merchants \\
6 & Wilds Outpost \\
\hline
\end{xtabular}
}
\end{center}

\subsection{Farming Village}

Your home was a farming village. There were at most a couple hundred
people living there. The surrounding land was mostly wilderness, but
there were a few other villages within a day's ride. Children were
expected to begin working almost as soon as they could walk. Life was
simple but hard, and you gained an appreciation for hard work and
an attitude of never taking things for granted.

\subsubsection{Farming Village Perks}

\begin{itemize}
  \item Skill Proficiency: Farming
  \item Equipment: A set of farmer's clothes
\end{itemize}

\subsection{Large City}

Your home was a large city, with all the bustling activity that comes
with it. Many different kinds of people found their home there, and
to some, it was chaotic. To those who lived there, though, it was a
wonderfully complex community. You didn't know everyone, but you knew
your own district, and you could always count on your neighbors for
help. The diversity of the city made you comfortable with outsiders.

\subsubsection{Large City Perks}

\begin{itemize}
  \item Skill Proficiency: Streetwise
  \item Equipment: A set of common clothes
\end{itemize}

\subsection{Nomad Tribe}

You grew up in a nomadic tribe. Your tribe moved from place to place,
living in tents, yurts, or other such mobile shelter. They didn't
always know where their next meal was coming from, but they usually
managed to find it together. The encroachment of civilization usually
meant having to pick up and move more often, but that didn't much
matter to you. Your tribe was always there for you, and you developed
a strong sense of loyalty.

\subsubsection{Nomad Tribe Perks}

\begin{itemize}
  \item Skill Proficiency: Survival
  \item Equipment: A set of nomad clothes
\end{itemize}

\subsection{Religious Order}

You grew up within the confines of a religious order. It was closeted
and cut off from the outside world. The order lived a simple, even
ascetic life, but never lacked for food. The pursuit of the truth and
a good and holy life was the most important aspect of the community.
You were instilled with a strong sense of right and wrong, and the
determination to uphold your faith above all else.

\subsubsection{Religious Order Perks}

\begin{itemize}
  \item Skill Proficiency: Religion
  \item Equipment: A holy symbol
\end{itemize}

\subsection{Traveling Merchants}

Growing up, your family was part of a large caravan of traveling merchants.
Your home was a wagon, and you never spent much time in any one place.
Your family's traveling companions changed occasionally, and you never
knew if the newcomers would be friends or troublemakers. You learned to
be cautious of others, but hide your true feelings behind a mask.

\subsubsection{Traveling Merchants Perks}

\begin{itemize}
  \item Skill Proficiency: Deception
  \item Equipment: A set of merchant's clothes
\end{itemize}

\subsection{Wilds Outpost}

Your home was a small fortified outpost on the remote edge of civilization.
Everyone living there was self-sufficient by necessity. As a frontier
outpost, it was more well-armed than a farming village, but there was a
reason for that. Your home suffered occasional attacks by monstrous
wildlife or even raiders. You grew up learning to fight and to take
care of yourself.

\subsubsection{Wilds Outpost Perks}

\begin{itemize}
  \item Skill Proficiency: Survival
  \item Equipment: A set of sturdy clothes
\end{itemize}

\section{List of Biomes}

Biomes describe the natural world where you grew up in. \index{biomes}

Choose one or roll 1d8 on the following table.

\bottomcaption{Random biomes}
\tablefirsthead{\hline \multicolumn{1}{|c|}{\textbf{Roll}} &
                       \multicolumn{1}{c|}{\textbf{Biome}} \\ \hline }
\begin{center}
{\rowcolors{2}{white}{light-gray}
\begin{xtabular}{|l|l|}
1 & Coast \\
2 & Desert \\
3 & Forest \\
4 & Jungle \\
5 & Mountains \\
6 & River \\
7 & Steppe \\
8 & Tundra \\
\hline
\end{xtabular}
}
\end{center}

\subsection{Coast}

Those that live on the coast tend towards bronze or tan complexions
and favor shells and other sea-borne items for decoration.

\subsubsection{Coast Skill Proficiencies}

\begin{itemize}
  \item Swimming
  \item Fishing
\end{itemize}

\subsection{Desert}

Desert dwellers have darker complexions, tend towards light but full
clothing, and are usually a little shorter.

\subsubsection{Desert Skill Proficiencies}

\begin{itemize}
  \item Heat Survival
  \item Find Water
\end{itemize}

\subsection{Forest}

People who live in or near forests have fairer skin, though darker
hair.

\subsubsection{Forest Skill Proficiencies}

\begin{itemize}
  \item Tracking
  \item Hiding
\end{itemize}

\subsection{Jungle}

Jungle folk have dark skin and often wear light and airy clothing.

\subsubsection{Jungle Skill Proficiencies}

\begin{itemize}
  \item Direction Sense
  \item Hunting
\end{itemize}

\subsection{Mountains}

People of the mountains are bigger and taller than their low-altitude
cousins. Their hair tends to be light in color, but their skin is heavily
tanned.

\subsubsection{Mountains Skill Proficiencies}

\begin{itemize}
  \item Climbing
  \item Cold Survival
\end{itemize}

\subsection{River}

Those who live beside rivers have a wide variety of skin and hair colors.
They also tend to be shorter than others.

\subsubsection{River Skill Proficiencies}

\begin{itemize}
  \item Swimming
  \item Fishing
\end{itemize}

\subsection{Steppe}

Steppe denizens are used to rolling hills and wide open spaces. They have
darker, reddish or bronzed skin and dark hair.

\subsubsection{Steppe Skill Proficiencies}

\begin{itemize}
  \item Riding
  \item Tracking
\end{itemize}

\subsection{Tundra}

Folk of the tundra live in bitter cold most of the year and are comfortable
in heavy snow and freezing rain. These people have fair skin and hair.

\subsubsection{Tundra Skill Proficiencies}

\begin{itemize}
  \item Cold Survival
  \item Snow Tracking
\end{itemize}

\end{multicols}
