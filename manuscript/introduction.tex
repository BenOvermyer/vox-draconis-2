\chapter{Introduction to Vox Draconis}

\begin{multicols}{2}

\section{About The Game}

\textit{Vox Draconis: Kingdoms of Stone and Fire} is my latest tabletop
role-playing game project. Thematically, it's based on my original \textit{Vox
Draconis} game. It's not set in the original game's
"Desova" setting, but it takes a few cues from that world. Dinosaurs
are as prevalent as other wildlife. The tech level is somewhere between
the Stone Age and Tolkienesque fantasy. The game system makes a few nods
to the original's, but is built around the player characters being true
heroes and not rampaging murderhobo tomb raiders.

"Treasure" as older tabletop role-playing games define it is much less
of a focus. Experience Points are awarded for having an impact on the
game world, and not explicitly for defeating monsters. It is perfectly
plausible to run a campaign where no combat takes place. Such a game
might focus on the social intrigues of a noble court, or perhaps the
misadventures of a group of street performers. This is up to you.

The game as a narrative device and set of social rules for a group of gamers 
also tries to support modern ideas of identity, gender, ancestry, and culture. 
The cultures in the setting may not adhere to these ideals, however. Components
of the setting that you are uncomfortable with should be modified or removed
from your own group's campaigns to suit.

\section{How It Works}

Vox Draconis requires at least two players, and works best with four or five. 
One person acts as the Game Master (GM), and the rest act as individual characters. The
GM is responsible for describing scenes for the others to participate in.
They are also responsible for making sure everyone is engaged in the game and
having fun.

To play Vox Draconis, the group will need some dice. Most of these are the common
six-sided dice you see everywhere. The rest have four sides, eight sides, ten sides,
twelve sides, or twenty sides. You can find them for sale online or in
local shops that specialize in tabletop games.

\section{Common Acronyms}

Through the text, there will be shortened references to various concepts.
The following is a guide to these acronyms.

\begin{itemize}
    \item GM: Game Master
    \item NPC: Non-Player Character
    \item AP: Armor Point
    \item XP: Experience Point
    \item PP: Player Point
    \item PS: Physical Strength
    \item PD: Physical Dexterity
    \item PE: Physical Endurance
    \item MS: Mental Strength
    \item MD: Mental Dexterity
    \item ME: Mental Endurance
    \item SS: Social Strength
    \item SD: Social Dexterity
    \item SE: Social Endurance
    \item AS: Arcane Strength
    \item AD: Arcane Dexterity
    \item AE: Arcane Endurance
\end{itemize}

\end{multicols}
